\documentclass{article}

\usepackage{arxiv}

\usepackage[utf8]{inputenc} % allow utf-8 input
\usepackage[T1]{fontenc}    % use 8-bit T1 fonts
\usepackage{hyperref}       % hyperlinks
\usepackage{url}            % simple URL typesetting
\usepackage{booktabs}       % professional-quality tables
\usepackage{amsfonts}       % blackboard math symbols
\usepackage{nicefrac}       % compact symbols for 1/2, etc.
\usepackage{microtype}      % microtypography
\usepackage{lipsum}		% Can be removed after putting your text content
\usepackage{graphicx}
\usepackage[numbers]{natbib}
\usepackage{doi}

\title{Lagrange-ng: The next generation of the DEC model}

\author{\href{https://orcid.org/0000-0002-9130-6878}{\includegraphics[scale=0.06]{orcid.pdf}\hspace{1mm}Ben Bettisworth}\thanks{} \\
  Computational Molecular Evolution\\
  Heidelberg Institute for Theoretical Studies\\
  Heidelberg, Germany \\
  \texttt{ben.bettisworth@h-its.org} \\
  %% examples of more authors
  \And
  Alexandros Stamatakis\\
  Computational Molecular Evolution\\
  Heidelberg Institute for Theoretical Studies\\
  Heidelberg, Germany\\
  \texttt{alexandros.stamatakis@h-its.org} \\
}

\renewcommand{\shorttitle}{\textit{arXiv} Template}

%%% Add PDF metadata to help others organize their library
%%% Once the PDF is generated, you can check the metadata with
%%% $ pdfinfo template.pdf
\hypersetup{
  pdftitle={A template for the arxiv style},
  pdfsubject={q-bio.NC, q-bio.QM},
  pdfauthor={Ben Bettisworth, Alexandros Stamatakis},
  pdfkeywords={},
}

\begin{document}
\maketitle

\begin{abstract}
\end{abstract}

\keywords{First keyword \and Second keyword \and More}

\section{Introduction}

\section{Background}

Computation of the likelihood of a particular set of parameters under the DEC model proceeds in a fashion similar to the
Felsenstein algorithm. In this algorithm, the computation starts from the tips and moves towards the root, storing
intermediate results in buffers called conditional likelihood vectors. Please see \cite{yang2006computational} for a
much more detailed explanation, including a detailed discussion about the savings involved with such a scheme. What is
relevant for the discussion here is that the Felsenstein algorithm avoids excess computation by noticing that, at
certain points in the computation of a likelihood on a tree, the only relevant number is the likelihood
\textit{conditioned on the current state}.

Typical phylogenetic likelihood computations have a computational cost that is dominated by simply updating the

\section{Methods and Algorithms}
\label{sec:methods}

\section{Experiments}
\label{sec:experiments}

\section{Results}
\label{sec:results}

\section{Discussion}
\label{sec:discussion}

\section{Conclusion}
\label{sec:conclusion}

\bibliographystyle{acm}
\bibliography{references}

\end{document}
